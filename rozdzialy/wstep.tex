\chapter{Wstęp}
Widzenie maszynowe jest podzbiorem szerokiej dziedziny informatyki jaką jest przetwarzanie obrazów. W ostatnich latach nauka ta dynamicznie się rozwija wraz ze wzrostem mocy obliczeniowej. W ramach widzenia maszynowego niezwykle ciekawym zagadnieniem jest rozpoznawanie obrazów. 

Samo zadanie rozpoznawania obrazów, nie jest z punktu widzenia sztucznej inteligencji dużym problem. Aby móc rozpoznawać obrazy należy posiadać zbiór cech według których będziemy klasyfikować. Przetwarzanie obrazów rozumiane jako filtracja i analiza, mające na celu opracowanie zbioru cech obrazu jest najbardziej złożonym  i skomplikowanym procesem. 

Początki widzenia maszynowego znajdują się w przemyśle, gdzie systemy wizyjne były i są wykorzystywane do nadzorowania procesów produkcyjnych. Odbywa się to w ściśle określonych i zoptymalizowanych dla konkretnego systemu warunkach działania. W raz rozwojem innych dziedzin informatyki takich jak np. rozszerzona rzeczywistość czy robotyka, widzeniu maszynowego stawiane są coraz to nowe wyzwania. Są to np. jak praca w zmiennej orientacji, w zróżnicowanym środowisku w dowolnych warunkach. Wymusza to poszukiwanie nowych rozwiązań wykrywających cechy charakterystyczne obiektów. Cechy takie powinny być możliwie jak najmniej zależne od transformacji takich jak skalowanie i rotacja obrazu,lokalizacji kamery oraz odporne na zmienne oświetlenie obiektu. 

Celem badawczym niniejszej pracy było przebadanie 6 algorytmów służących do ekstrakcji i opisu cech obrazów. Algorytmy te zostały przetestowane pod kątem skuteczności jak i szybkości działania. Ponadto podjęto prace mające na celu opracowanie hybrydowego algorytmu będącego syntezą wcześniej opisanych metod metod. Jako przykład skomplikowanych form przestrzennych zdecydowano się część testów wykonać na obrazach przedstawiających formacje skalne.