\chapter{Opis badań}
Omówione w części teoretycznej algorytmy zostały przebadane pod kątem efektywności działania.
W tym celu został napisany program komputerowy, którego szerszy opis można znaleźć w dodatku \ref{dodatek_opis_aplikacji}.
\FloatBarrier 
\section{Badane algorytmy}
Problem poprawnego rozpoznawania obrazów opartego o cechy charakterystyczne można zdekomponować na dwa mniejsze jakimi są:
\begin{itemize}
\item lokalizowanie punktów charakterystycznych,
\item opisywanie punktów charakterystycznych.
\end{itemize}
Oba aspekty są kluczowe dla realizacji postawionego zadania, a efekt finalny jest wypadkową pracy detektorów i deskryptorów. Z tego względu przyjęto założenie, że algorytmy kompletne tj. zawierające zarówno detektor jak i deskryptor zostaną zestawione z algorytmami mieszanymi. W przypadku algorytmów nie posiadających własnych mechanizmów opisu cech, pulę dostępnych deskryptorów wzbogacono o metody algorytmów kompletnych. Zastosowane podejście pozwoliło porównać ze sobą niezależnie stosowane detektory i deskryptory.
Postanowiono przebadać następujące algorytmy:
\begin{itemize}
\item ORB
\item SIFT
\item SURF
\item detektor STAR + deskryptor BRIEF
\item detektor STAR + deskryptor ORB (\textit{wł. rBRIEF})
\item detektor STAR + deskryptor SIFT
\item detektor STAR + deskryptor SURF
\item detektor FAST + deskryptor BRIEF $^1$
\item detektor FAST + deskryptor ORB (\textit{wł. rBRIEF}) $^1$
\item detektor FAST + deskryptor SIFT $^1$
\item detektor FAST + deskryptor SURF \footnote{Pierwsze testy wykazały jednakże, że ilość punktów charakterystycznych lokalizowanych przez algorytm FAST jest zbyt duża, aby można było dokonać dopasowania w rozsądnym czasie pracy maszyny.W związku z powyższym zrezygnowano z opisywania i dopasowywania cech lokalizowanych przez ten algorytm}
\end{itemize}


\FloatBarrier
\section{Kryteria badań}
W porównywaniu algorytmów wzięto pod uwagę następujące kryteria:
\begin{itemize}
\item ilość lokalizowanych cech
\item czas potrzebny na zlokalizowanie pojedynczego punktu charakterystycznego
\item czas potrzebny na wygenerowanie deskryptora pojedynczego punktu charakterystycznego
\item powtarzalność zlokalizowanych cech
\item procent poprawnych dopasowań
\end{itemize}



Ze względu na to, że algorytmy różnią się czasem działania i ilością lokalizowanych cech w rożnych obrazach, wszystkie kryteria czasowe zdecydowano się odnieść do pojedynczego punktu.


Powtarzalność zlokalizowanych cech jest rozumiana jako:
\begin{equation}
p_{AB}  = \frac{d_{AB}}{i_{AB}}
\end{equation}
gdzie:\\
$p_{AB}$ - powtarzalność lokalizowanych cech między obrazami A i B\\
$d_{AB}$ - ilość dopasowanych cech między obrazami A i B metodą k najbliższych sąsiadów\\
$i_{AB}$ - średnia arytmetyczna ilości punktów zlokalizowanych w obrazach A i B\\


Do oceny jakości parowania cech wykorzystano algorytm RANSAC \cite{ransac}. Pozwala on wyeliminować błędne dopasowania nie pasujące do estymowanego wzorca. Za procent poprawnych dopasowań rozumiemy stosunek ilość par punktów po zastosowaniu algorytmu RANSAC do ilości wstępnie dopasowanych punktów.
\FloatBarrier
\section{Zbiory testowe}
Do oceny pracy algorytmów wykorzystano dwa zbiory testowe:
\begin{itemize}
\item zbiór testowy Mikołajczyka
\item autorski zbiór testowy przedstawiający obrazy skalne
\end{itemize}
W skład obu zbiorów wchodzą zestawy obrazów pogrupowane wg, przekształceń takich jak rotacja, zmiana skali, zmiana oświetlenie lub położenia obserwatora. Każdy pojedynczy zestaw składa się z 6 zdjęć. Pierwsze zdjęcie jest obiektem referencyjnym względem którego dokonywane są dopasowania. Kolejne zdjęcia zestawu są poddawane coraz mocniejszym przekształceniom.

\FloatBarrier