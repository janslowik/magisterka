\chapter{Zbiór testowy Mikołajczyka}
Zbiór testowy Mikołajczyka to opracowany przez doktora Krystiana Mikołajczyka zbiór zestawów obrazów testowych służących ocenie afinicznych, kowariantnych cech. Zawiera on przekształcenia takie jak:
\begin{itemize}
\item zmiana położenia obserwatora
\item zmiana skali 
\item rotacja
\item zmiana oświetlenia
\item rozmycie
\item kompresja JPEG
\end{itemize}
Oryginalny zbiór zawiera 8 zestawów zawierające powyższe transformacje. Zbiory dotyczące położenia obserwatora, skali i rozmycia są zdublowane. Uczyniono to aby uniezależnić ocenę jakości pracy z  przekształcenie od charakterystyki obrazu. W przeprowadzonych badaniach zrezygnowano z testowania odporności algorytmów na kompresje JPEG. Ze względu na fakt, że zbiór Mikołajczyka jest często wykorzystywany w badaniach, zdecydowano się zachować angielskie nazewnictwo zestawów.
\FloatBarrier
\newpage
\section{Rozmycie obrazu}
\FloatBarrier
\subsection{BIKES}

\begin{figure}[!htb]
\begin{center}

\subfigure[BIKES 1]{
\includegraphics[width=5cm]{pict/mikolajczyk/bikes/img1.jpg}
}
\subfigure[BIKES 2]{
\includegraphics[width=5cm]{pict/mikolajczyk/bikes/img2.jpg}
}
\subfigure[BIKES 3]{
\includegraphics[width=5cm]{pict/mikolajczyk/bikes/img3.jpg}
}
\subfigure[BIKES 4]{
\includegraphics[width=5cm]{pict/mikolajczyk/bikes/img4.jpg}
}
\subfigure[BIKES 5]{
\includegraphics[width=5cm]{pict/mikolajczyk/bikes/img5.jpg}
}
\subfigure[BIKES 6]{
\includegraphics[width=5cm]{pict/mikolajczyk/bikes/img6.jpg}
}
\caption{BIKES - zbiór obrazów testowych}
\label{fig:bikes_set}
\end{center}
\end{figure}



% Table generated by Excel2LaTeX from sheet 'm.bikes F'
\begin{table}[htbp]
  \centering
  \caption{BIKES - ilość wyszukanych cech}
    \begin{tabular}{|c|r|r|r|r|r|}\hline
    
    obraz & \textbf{ORB} & \textbf{SIFT} & \textbf{SURF} & \textbf{STAR} & \textbf{FAST} \\\hline
    
   
    1 & 500 & 1014 & 12911 & 586 & 10702 \\
    2 & 500 & 1062 & 8974 & 362 & 3128 \\
    3 & 500 & 1159 & 8100 & 290 & 2247 \\
    4 & 485 & 839 & 7151 & 181 & 1129 \\
    5 & 420 & 596 & 6344 & 139 & 675 \\
    6 & 366 & 473 & 5751 & 118 & 500 \\\hline
    \textbf{średnia} & \textbf{462} & \textbf{857} & \textbf{8205} & \textbf{279} & \textbf{3064} \\
    \hline
    \end{tabular}%
  \label{tab:bikes_f1}%
\end{table}%


\begin{figure}
\centering
\includegraphics[width=0.8\textwidth]{pict/mikolajczyk/bikes/F1.png}
\caption{BIKES - ilość wyszukanych cech}
\label{fig:bikes_f1}
\end{figure}


% Table generated by Excel2LaTeX from sheet 'm.bikes F'
\begin{table}[htbp]
  \centering
  \caption{BIKES - czas lokalizowania pojedynczego punktu charakterystycznego}
    \begin{tabular}{|c|c|c|c|c|c|}
    \hline
    obraz & \textbf{ORB} & \textbf{SIFT} & \textbf{SURF} & \textbf{STAR} & \textbf{FAST} \\
    \hline
   
    1 & 0,080 & 0,179 & 0,156 & 0,109 & 0,001 \\
    2 & 0,062 & 0,171 & 0,169 & 0,182 & 0,002 \\
    3 & 0,060 & 0,153 & 0,173 & 0,217 & 0,002 \\
    4 & 0,056 & 0,204 & 0,179 & 0,348 & 0,003 \\
    5 & 0,060 & 0,282 & 0,185 & 0,453 & 0,003 \\
    6 & 0,066 & 0,340 & 0,191 & 0,542 & 0,004 \\\hline
    \textbf{średnia} & \textbf{0,064} & \textbf{0,222} & \textbf{0,175} & \textbf{0,309} & \textbf{0,002} \\
   \hline
    \end{tabular}%
  \label{tab:bikes_f2}%
\end{table}%


\begin{figure}
\centering
\includegraphics[width=0.8\textwidth]{pict/mikolajczyk/bikes/f2.png}
\caption{BIKES - czas lokalizowania pojedynczego punktu charakterystycznego}
\end{figure}

% Table generated by Excel2LaTeX from sheet 'm.bikes F'
\begin{table}[htbp]
  \centering
  \caption{BIKES - czas generowania pojedynczego deskryptora punktu charakterystycznego}
    \begin{tabular}{|c|c|c|c|c|c|c|c|}\hline

    obraz & \textbf{ORB} & \textbf{SIFT} & \textbf{SURF} & \textbf{ST-BRIEF} & \textbf{ST-ORB} & \textbf{ST-SIFT} & \textbf{ST-SURF} \\\hline

    - & [ms] & [ms] & [ms] & [ms] & [ms] & [ms] & [ms] \\\hline
    1 & 0,092 & 0,300 & 0,193 & 0,015 & 0,019 & 1,109 & 0,094 \\
    2 & 0,090 & 0,293 & 0,201 & 0,019 & 0,025 & 1,561 & 0,099 \\
    3 & 0,090 & 0,283 & 0,199 & 0,017 & 0,031 & 1,803 & 0,103 \\
    4 & 0,093 & 0,344 & 0,194 & 0,022 & 0,044 & 2,420 & 0,116 \\
    5 & 0,107 & 0,424 & 0,195 & 0,029 & 0,050 & 3,201 & 0,129 \\
    6 & 0,120 & 0,474 & 0,198 & 0,025 & 0,059 & 3,873 & 0,136 \\\hline
    \textbf{średnia} & \textbf{0,099} & \textbf{0,353} & \textbf{0,197} & \textbf{0,021} & \textbf{0,038} & \textbf{2,328} & \textbf{0,113} \\\hline
    

    \end{tabular}%
  \label{tab:bikes_f3}%
\end{table}%


\begin{figure}
\centering
\includegraphics[width=0.8\textwidth]{pict/mikolajczyk/bikes/f3.png}
\caption{BIKES - czas generowania pojedynczego deskryptora punktu charakterystycznego}
\label{fig:bikes_f3}
\end{figure}


% Table generated by Excel2LaTeX from sheet 'm.bikes M'
\begin{table}[htbp]
  \centering
  \caption{BIKES - powtarzalność wykrywanych cech}
    \begin{tabular}{|c|c|c|c|c|c|c|c|}\hline

    obrazy & \textbf{ORB} & \textbf{SIFT} & \textbf{SURF} & \textbf{ST-BRIEF} & \textbf{ST-ORB} & \textbf{ST-SIFT} & \textbf{ST-SURF} \\\hline

    -  & [\%] & [\%] & [\%] & [\%] & [\%] & [\%] & [\%] \\\hline
    1|2 & 44 & 50 & 38 & 55 & 59 & 61 & 27 \\
    1|3 & 38 & 39 & 25 & 43 & 42 & 42 & 20 \\
    1|4 & 23 & 36 & 16 & 32 & 28 & 30 & 11 \\
    1|5 & 18 & 32 & 11 & 35 & 22 & 33 & 8 \\
    1|6 & 12 & 27 & 7 & 19 & 19 & 29 & 5 \\\hline
    \textbf{średnia} & \textbf{27} & \textbf{37} & \textbf{19} & \textbf{37} & \textbf{34} & \textbf{39} & \textbf{14} \\\hline
    

    \end{tabular}%
  \label{tab:bikes_m1}%
\end{table}%


\begin{figure}
\centering
\includegraphics[width=0.8\textwidth]{pict/mikolajczyk/bikes/m1.png}
\caption{BIKES - powtarzalność wykrywanych cech}
\label{fig:bikes_m1}
\end{figure}

% Table generated by Excel2LaTeX from sheet 'm.bikes M'
\begin{table}[htbp]
  \centering
  \caption{BIKES - procent poprawnych dopasowań}
    \begin{tabular}{|c|c|c|c|c|c|c|c|}\hline
    obrazy & \textbf{ORB} & \textbf{SIFT} & \textbf{SURF} & \textbf{ST-BRIEF} & \textbf{ST-ORB} & \textbf{ST-SIFT} & \textbf{ST-SURF} \\\hline
     - & [\%] & [\%] & [\%] & [\%] & [\%] & [\%] & [\%] \\\hline
   1|2 & 64 & 88 & 66 & 59 & 65 & 73 & 59 \\
    1|3 & 58 & 74 & 52 & 54 & 51 & 55 & 57 \\
    1|4 & 37 & 62 & 50 & 32 & 40 & 41 & 39 \\
    1|5 & 41 & 69 & 31 & 38 & 32 & 40 & 39 \\
    1|6 & 44 & 49 & 30 & 23 & 28 & 37 & 32 \\\hline
    \textbf{średnia} & \textbf{49} & \textbf{68} & \textbf{46} & \textbf{41} & \textbf{43} & \textbf{49} & \textbf{45} \\\hline
    
    \end{tabular}%
  \label{tab:bikes_m2}%
\end{table}%


\begin{figure}
\centering
\includegraphics[width=0.8\textwidth]{pict/mikolajczyk/bikes/m2.png}
\caption{BIKES - procent poprawnych dopasowań}
\label{fig:bikes_m2}
\end{figure}








\newpage
\FloatBarrier
\subsection{TREES}

\begin{figure}[!htb]
\begin{center}
\subfigure[TREES 1]{
\includegraphics[width=5cm]{pict/mikolajczyk/trees/img1.jpg}
}
\subfigure[TREES 2]{
\includegraphics[width=5cm]{pict/mikolajczyk/trees/img2.jpg}
}
\subfigure[TREES 3]{
\includegraphics[width=5cm]{pict/mikolajczyk/trees/img3.jpg}
}
\subfigure[TREES 4]{
\includegraphics[width=5cm]{pict/mikolajczyk/trees/img4.jpg}
}
\subfigure[TREES 5]{
\includegraphics[width=5cm]{pict/mikolajczyk/trees/img5.jpg}
}
\subfigure[TREES 6]{
\includegraphics[width=5cm]{pict/mikolajczyk/trees/img6.jpg}
}
\caption{TREES - zbiór obrazów testowych}
\label{fig:trees_set}
\end{center}
\end{figure}
\FloatBarrier


% Table generated by Excel2LaTeX from sheet 'm.trees F'
\begin{table}[htbp]
  \centering
  \caption{TREES - ilość wyszukanych cech}
    \begin{tabular}{|c|r|r|r|r|r|}\hline
    
    obraz & \textbf{ORB} & \textbf{SIFT} & \textbf{SURF} & \textbf{STAR} & \textbf{FAST} \\\hline
    
   
    1 & 500 & 2941 & 11343 & 2539 & 34024 \\
    2 & 500 & 2848 & 12563 & 2536 & 38235 \\
    3 & 500 & 3363 & 10377 & 2846 & 28516 \\
    4 & 500 & 4150 & 8613 & 2398 & 16582 \\
    5 & 500 & 4244 & 8022 & 1383 & 11716 \\
    6 & 500 & 2709 & 6605 & 824 & 7883 \\\hline
    \textbf{średnia} & \textbf{500} & \textbf{3376} & \textbf{9587} & \textbf{2088} & \textbf{22826} \\
   \hline
    \end{tabular}%
  \label{tab:trees_f1}%
\end{table}%


\begin{figure}
\centering
\includegraphics[width=0.8\textwidth]{pict/mikolajczyk/trees/F1.png}
\caption{TREES - ilość wyszukanych cech}
\label{fig:trees_f1}
\end{figure}


% Table generated by Excel2LaTeX from sheet 'm.trees F'
\begin{table}[htbp]
  \centering
  \caption{TREES - czas lokalizowania pojedynczego punktu charakterystycznego w obrazach z zestawu TREES}
    \begin{tabular}{|c|c|c|c|c|c|}
    \hline
    obraz & \textbf{ORB} & \textbf{SIFT} & \textbf{SURF} & \textbf{STAR} & \textbf{FAST} \\
    \hline
    -  & [ms] & [ms] & [ms] & [ms] & [ms] \\\hline
    1 & 0,162 & 0,077 & 0,162 & 0,027 & 0,001 \\
    2 & 0,172 & 0,088 & 0,158 & 0,026 & 0,001 \\
    3 & 0,158 & 0,067 & 0,165 & 0,024 & 0,001 \\
    4 & 0,130 & 0,058 & 0,173 & 0,028 & 0,002 \\
    5 & 0,102 & 0,058 & 0,175 & 0,047 & 0,002 \\
    6 & 0,078 & 0,085 & 0,185 & 0,078 & 0,001 \\\hline
    \textbf{średnia} & \textbf{0,134} & \textbf{0,072} & \textbf{0,170} & \textbf{0,038} & \textbf{0,001} \\
   \hline
    \end{tabular}%
  \label{tab:trees_f2}%
\end{table}%


\begin{figure}
\centering
\includegraphics[width=0.8\textwidth]{pict/mikolajczyk/trees/f2.png}
\caption{TREES - czas lokalizowania pojedynczego punktu charakterystycznego}
\label{fig:trees_f2}
\end{figure}

% Table generated by Excel2LaTeX from sheet 'm.trees F'
\begin{table}[htbp]
  \centering
  \caption{TREES - czas generowania pojedynczego deskryptora punktu charakterystycznego}
    \begin{tabular}{|c|c|c|c|c|c|c|c|}\hline

    obraz & \textbf{ORB} & \textbf{SIFT} & \textbf{SURF} & \textbf{ST-BRIEF} & \textbf{ST-ORB} & \textbf{ST-SIFT} & \textbf{ST-SURF} \\\hline

    - & [ms] & [ms] & [ms] & [ms] & [ms] & [ms] & [ms] \\\hline
    1 & 0,064 & 0,207 & 0,209 & 0,012 & 0,012 & 0,460 & 0,083 \\
    2 & 0,064 & 0,210 & 0,203 & 0,012 & 0,012 & 0,442 & 0,084 \\
    3 & 0,064 & 0,197 & 0,209 & 0,012 & 0,011 & 0,516 & 0,084 \\
    4 & 0,064 & 0,188 & 0,213 & 0,012 & 0,012 & 0,620 & 0,085 \\
    5 & 0,064 & 0,200 & 0,204 & 0,012 & 0,014 & 1,091 & 0,093 \\
    6 & 0,064 & 0,240 & 0,202 & 0,013 & 0,017 & 1,602 & 0,103 \\\hline
    \textbf{średnia} & \textbf{0,064} & \textbf{0,207} & \textbf{0,207} & \textbf{0,012} & \textbf{0,013} & \textbf{0,789} & \textbf{0,089} \\\hline
    

    \end{tabular}%
  \label{tab:trees_f3}%
\end{table}%


\begin{figure}
\centering
\includegraphics[width=0.8\textwidth]{pict/mikolajczyk/trees/f3.png}
\caption{TREES - czas generowania pojedynczego deskryptora punktu charakterystycznego}
\label{fig:trees_f3}
\end{figure}


% Table generated by Excel2LaTeX from sheet 'm.trees M'
\begin{table}[htbp]
  \centering
  \caption{TREES - powtarzalność wykrywanych cech}
    \begin{tabular}{|c|c|c|c|c|c|c|c|}\hline

    obrazy & \textbf{ORB} & \textbf{SIFT} & \textbf{SURF} & \textbf{ST-BRIEF} & \textbf{ST-ORB} & \textbf{ST-SIFT} & \textbf{ST-SURF} \\\hline

    -  & [\%] & [\%] & [\%] & [\%] & [\%] & [\%] & [\%] \\\hline
    1|2 & 18 & 18 & 11 & 38 & 12 & 38 & 3 \\
    1|3 & 25 & 15 & 9 & 32 & 9 & 32 & 2 \\
    1|4 & 13 & 8 & 4 & 20 & 4 & 18 & 1 \\
    1|5 & 6 & 5 & 3 & 18 & 5 & 9 & 1 \\
    1|6 & 3 & 4 & 1 & 14 & 2 & 3 & 0 \\\hline
    \textbf{średnia} & \textbf{13} & \textbf{10} & \textbf{6} & \textbf{24} & \textbf{6} & \textbf{20} & \textbf{1} \\\hline
   

    \end{tabular}%
  \label{tab:trees_m1}%
\end{table}%


\begin{figure}
\centering
\includegraphics[width=0.8\textwidth]{pict/mikolajczyk/trees/m1.png}
\caption{TREES - powtarzalność wykrywanych cech}
\label{fig:trees_m1}
\end{figure}

% Table generated by Excel2LaTeX from sheet 'm.trees M'
\begin{table}[htbp]
  \centering
  \caption{TREES - procent poprawnych dopasowań}
    \begin{tabular}{|c|c|c|c|c|c|c|c|}\hline
    obrazy & \textbf{ORB} & \textbf{SIFT} & \textbf{SURF} & \textbf{ST-BRIEF} & \textbf{ST-ORB} & \textbf{ST-SIFT} & \textbf{ST-SURF} \\\hline
     - & [\%] & [\%] & [\%] & [\%] & [\%] & [\%] & [\%] \\\hline
    1|2 & 43 & 45 & 50 & 56 & 71 & 62 & 33 \\
    1|3 & 71 & 47 & 40 & 63 & 62 & 67 & 30 \\
    1|4 & 57 & 35 & 28 & 41 & 35 & 51 & 16 \\
    1|5 & 31 & 30 & 18 & 41 & 23 & 38 & 6 \\
    1|6 & 33 & 23 & 13 & 33 & 14 & 20 & 4 \\\hline
    \textbf{średnia} & \textbf{47} & \textbf{36} & \textbf{30} & \textbf{47} & \textbf{41} & \textbf{48} & \textbf{18} \\\hline
    
    \end{tabular}%
  \label{tab:trees_m2}%
\end{table}%


\begin{figure}
\centering
\includegraphics[width=0.8\textwidth]{pict/mikolajczyk/trees/m2.png}
\caption{TREES - procent poprawnych dopasowań}
\label{fig:trees_m2}
\end{figure}
\FloatBarrier
\subsection{Dyskusja wyników}
Jak możemy zaobserwować na wykresach \ref{fig:bikes_f1} i \ref{fig:tree_f1} wraz ze wzrostem stopnia rozmycia obrazu maleje liczba lokalizowanych punktów charakterystycznych. Wśród ilości lokalizowanych punktów prym wiodą w zależności od scenerii algorytmy SURF i FAST, i to wśród ich wyników spadek ilości punktów jest najbardziej zauważalny. Algorytmy te generują o rząd wielkości więcej punktów. Niestety jakość tych punktów pozostawia często wiele do życzenia, a ich ilość w sposób znaczący utrudnia proces dopasowywania. Rysunek \ref{tt13} przedstawia punkty wyszukane przez algorytm FAST w obrazach z zestawu TREES. Jak widzimy ich ilość w sposób znaczący zaciemnia obraz.

Większość algorytmów jest wrażliwa na stopień rozmycia, co przekłada się na czas lokalizowania punktów charakterystycznych. Na tym tle korzystnie wypada algorytm ORB, poprawiający swoje charakterystyki czasowe przy zachowanej stałej liczbie lokalizowanych punktów.

W przypadku generowania deskryptora punktu charakterystycznego większość algorytmów zachowuje względnie stałe wartości czasowe. Wyniki te mogą się nieznacznie różnić w zależności od analizowanej sceny. Wyjątkiem tu są dokładne deskryptory gradientowe SIFT. Wraz ze wzrostem rozmycia obserwujemy wzrost czasu potrzebnego na obliczenie deskryptora. Szczególnie dramatyczny wzrost ma miejsce w przypadku kombinacji deskryptora SIFT z detektorem STAR. Częściową tą wadę rekompensuje wysoka na tle innych algorytmów powtarzalność dopasowań.

Powtarzalność dopasowań dla wszystkich algorytmów spada wraz ze wzrostem rozmycia. Parametr ten w zależności od scenerii osiąga maksymalną wartość od 30 do 60 \%. Jak pokazują algorytmy są wrażliwe w przypadku rozmycia obrazów o dużej ilości drobnych, podobnych elementów jakimi są liście.

W przypadku obrazów o zróżnicowanych elementach najlepszy współczynnik trafnych dopasowań osiąga algorytm SIFT. W sytuacji gdy obraz wskutek rozmycia w znacznym stopniu staje się nieczytelny jak w przypadku obrazów TREES lepszą odpornością wykazują się algorytm ORB i kombinacja algorytmów STAR i BRIEF.







\begin{figure}
\centering
\includegraphics[width=0.8\textwidth]{pict/badania/trees_fast_1_3.png}
\caption{Zlokalizowane przez algorytm FAST punkty charakterystyczne w obrazach TREES 1 i TREES 3}
\label{tt13}
\end{figure}





\FloatBarrier
\section{Zmiana położenia obserwatora}
\FloatBarrier
\subsection{GRAFFITI}

\begin{figure}[!htb]
\begin{center}
\subfigure[GRAFFITI 1]{
\includegraphics[width=5cm]{pict/mikolajczyk/graff/img1.jpg}
}
\subfigure[GRAFFITI 2]{
\includegraphics[width=5cm]{pict/mikolajczyk/graff/img2.jpg}
}
\subfigure[GRAFFITI 3]{
\includegraphics[width=5cm]{pict/mikolajczyk/graff/img3.jpg}
}
\subfigure[GRAFFITI 4]{
\includegraphics[width=5cm]{pict/mikolajczyk/graff/img4.jpg}
}
\subfigure[GRAFFITI 5]{
\includegraphics[width=5cm]{pict/mikolajczyk/graff/img5.jpg}
}
\subfigure[GRAFFITI 6]{
\includegraphics[width=5cm]{pict/mikolajczyk/graff/img6.jpg}
}
\caption{GRAFFITI - zbiór obrazów testowych}
\label{fig:graffiti_set}
\end{center}
\end{figure}
\FloatBarrier


% Table generated by Excel2LaTeX from sheet 'm.graff F'
\begin{table}[htbp]
  \centering
  \caption{GRAFFITI - ilość wyszukanych cech}
    \begin{tabular}{|c|r|r|r|r|r|}\hline
    
    obraz & \textbf{ORB} & \textbf{SIFT} & \textbf{SURF} & \textbf{STAR} & \textbf{FAST} \\\hline
    
   
    1 & 500 & 1128 & 8444 & 878 & 6727 \\
    2 & 500 & 1296 & 8538 & 930 & 7567 \\
    3 & 500 & 1358 & 8869 & 1147 & 8579 \\
    4 & 500 & 1445 & 8888 & 973 & 11146 \\
    5 & 500 & 1500 & 9165 & 1164 & 10053 \\
    6 & 500 & 1478 & 9875 & 910 & 13459 \\\hline
    \textbf{średnia} & \textbf{500} & \textbf{1368} & \textbf{8963} & \textbf{1000} & \textbf{9589} \\
    \hline
    \end{tabular}%
  \label{tab:graffiti_f1}%
\end{table}%


\begin{figure}
\centering
\includegraphics[width=0.8\textwidth]{pict/mikolajczyk/graff/F1.png}
\caption{GRAFFITI - ilość wyszukanych cech}
\end{figure}


% Table generated by Excel2LaTeX from sheet 'm.graff F'
\begin{table}[htbp]
  \centering
  \caption{GRAFFITI - czas lokalizowania pojedynczego punktu charakterystycznego}
    \begin{tabular}{|c|c|c|c|c|c|}
    \hline
    obraz & \textbf{ORB} & \textbf{SIFT} & \textbf{SURF} & \textbf{STAR} & \textbf{FAST} \\
    \hline
    -  & [ms] & [ms] & [ms] & [ms] & [ms] \\\hline
    1 & 0,074 & 0,122 & 0,158 & 0,054 & 0,001 \\
    2 & 0,080 & 0,113 & 0,158 & 0,051 & 0,001 \\
    3 & 0,088 & 0,108 & 0,156 & 0,041 & 0,001 \\
    4 & 0,092 & 0,102 & 0,156 & 0,049 & 0,001 \\
    5 & 0,088 & 0,099 & 0,156 & 0,041 & 0,001 \\
    6 & 0,098 & 0,101 & 0,153 & 0,052 & 0,001 \\\hline
    \textbf{średnia} & \textbf{0,087} & \textbf{0,108} & \textbf{0,156} & \textbf{0,048} & \textbf{0,001} \\
    \hline
    \end{tabular}%
  \label{tab:graffiti_f2}%
\end{table}%


\begin{figure}
\centering
\includegraphics[width=0.8\textwidth]{pict/mikolajczyk/graff/f2.png}
\caption{GRAFFITI - czas lokalizowania pojedynczego punktu charakterystycznego}
\end{figure}

% Table generated by Excel2LaTeX from sheet 'm.graff F'
\begin{table}[htbp]
  \centering
  \caption{GRAFFITI - czas generowania pojedynczego deskryptora punktu charakterystycznego}
    \begin{tabular}{|c|c|c|c|c|c|c|c|}\hline

    obraz & \textbf{ORB} & \textbf{SIFT} & \textbf{SURF} & \textbf{ST-BRIEF} & \textbf{ST-ORB} & \textbf{ST-SIFT} & \textbf{ST-SURF} \\\hline

    - & [ms] & [ms] & [ms] & [ms] & [ms] & [ms] & [ms] \\\hline
    1 & 0,070 & 0,269 & 0,180 & 0,013 & 0,014 & 1,264 & 0,118 \\
    2 & 0,068 & 0,255 & 0,177 & 0,013 & 0,014 & 1,372 & 0,097 \\
    3 & 0,068 & 0,244 & 0,179 & 0,012 & 0,013 & 1,261 & 0,095 \\
    4 & 0,070 & 0,239 & 0,172 & 0,012 & 0,013 & 1,209 & 0,095 \\
    5 & 0,068 & 0,235 & 0,185 & 0,012 & 0,013 & 1,062 & 0,090 \\
    6 & 0,070 & 0,235 & 0,180 & 0,012 & 0,014 & 1,031 & 0,091 \\\hline
    \textbf{średnia} & \textbf{0,069} & \textbf{0,246} & \textbf{0,179} & \textbf{0,012} & \textbf{0,014} & \textbf{1,200} & \textbf{0,098} \\\hline
   
    \end{tabular}%
  \label{tab:graffiti_f3}%
\end{table}%


\begin{figure}
\centering
\includegraphics[width=0.8\textwidth]{pict/mikolajczyk/graff/f3.png}
\caption{GRAFFITI - czas generowania pojedynczego deskryptora punktu charakterystycznego}
\end{figure}


% Table generated by Excel2LaTeX from sheet 'm.graff M'
\begin{table}[htbp]
  \centering
  \caption{GRAFFITI - powtarzalność wykrywanych cech}
    \begin{tabular}{|c|c|c|c|c|c|c|c|}\hline

    obrazy & \textbf{ORB} & \textbf{SIFT} & \textbf{SURF} & \textbf{ST-BRIEF} & \textbf{ST-ORB} & \textbf{ST-SIFT} & \textbf{ST-SURF} \\\hline

    -  & [\%] & [\%] & [\%] & [\%] & [\%] & [\%] & [\%] \\\hline
    - & [\%] & [\%] & [\%] & [\%] & [\%] & [\%] & [\%] \\
    1|2 & 31 & 30 & 16 & 6 & 4 & 6 & 11 \\
    1|3 & 7 & 16 & 7 & 4 & 4 & 4 & 3 \\
    1|4 & 1 & 2 & 1 & 3 & 2 & 1 & 3 \\
    1|5 & 0 & 1 & 2 & 2 & 2 & 1 & 2 \\
    1|6 & 0 & 0 & 2 & 2 & 2 & 0 & 2 \\\hline
    \textbf{średnia} & \textbf{8} & \textbf{10} & \textbf{6} & \textbf{3} & \textbf{3} & \textbf{2} & \textbf{4} \\\hline
    
    \end{tabular}%
  \label{tab:graffiti_m1}%
\end{table}%


\begin{figure}
\centering
\includegraphics[width=0.8\textwidth]{pict/mikolajczyk/graff/m1.png}
\caption{GRAFFITI - powtarzalność wykrywanych cech}
\end{figure}

% Table generated by Excel2LaTeX from sheet 'm.graff M'
\begin{table}[htbp]
  \centering
  \caption{GRAFFITI - procent poprawnych dopasowań}
    \begin{tabular}{|c|c|c|c|c|c|c|c|}\hline
    obrazy & \textbf{ORB} & \textbf{SIFT} & \textbf{SURF} & \textbf{ST-BRIEF} & \textbf{ST-ORB} & \textbf{ST-SIFT} & \textbf{ST-SURF} \\\hline
     - & [\%] & [\%] & [\%] & [\%] & [\%] & [\%] & [\%] \\\hline
    1|2 & 53 & 55 & 34 & 38 & 12 & 47 & 28 \\
    1|3 & 30 & 34 & 19 & 18 & 10 & 33 & 13 \\
    1|4 & 24 & 17 & 4 & 16 & 13 & 40 & 9 \\
    1|5 & 0 & 16 & 2 & 18 & 13 & 31 & 7 \\
    1|6 & 0 & 19 & 2 & 26 & 17 & 0 & 5 \\\hline
    \textbf{średnia} & \textbf{21} & \textbf{28} & \textbf{12} & \textbf{23} & \textbf{13} & \textbf{30} & \textbf{12} \\\hline
   
    \end{tabular}%
  \label{tab:graffiti_m2}%
\end{table}%


\begin{figure}
\centering
\includegraphics[width=0.8\textwidth]{pict/mikolajczyk/graff/m2.png}
\caption{GRAFFITI - procent poprawnych dopasowań}
\end{figure}




\FloatBarrier
\subsection{WALL}

\begin{figure}[!htb]
\begin{center}
\subfigure[WALL 1]{
\includegraphics[width=5cm, height=4cm]{pict/mikolajczyk/wall/img1.jpg}
}
\subfigure[WALL 2]{
\includegraphics[width=5cm, height=4cm]{pict/mikolajczyk/wall/img2.jpg}
}
\subfigure[WALL 3]{
\includegraphics[width=5cm, height=4cm]{pict/mikolajczyk/wall/img3.jpg}
}
\subfigure[WALL 4]{
\includegraphics[width=5cm, height=4cm]{pict/mikolajczyk/wall/img4.jpg}
}
\subfigure[WALL 5]{
\includegraphics[width=5cm, height=4cm]{pict/mikolajczyk/wall/img5.jpg}
}
\subfigure[WALL 6]{
\includegraphics[width=5cm, height=4cm]{pict/mikolajczyk/wall/img6.jpg}
}
\caption{WALL - zbiór obrazów testowych}
\label{fig:wall_set}
\end{center}
\end{figure}


% Table generated by Excel2LaTeX from sheet 'm.wall F'
\begin{table}[htbp]
  \centering
  \caption{WALL - ilość wyszukanych cech}
    \begin{tabular}{|c|r|r|r|r|r|}\hline
    
    obraz & \textbf{ORB} & \textbf{SIFT} & \textbf{SURF} & \textbf{STAR} & \textbf{FAST} \\\hline
    
    1 & 500 & 1821 & 14494 & 1461 & 36935 \\
    2 & 500 & 1870 & 11050 & 1434 & 27704 \\
    3 & 500 & 1840 & 10763 & 1330 & 26384 \\
    4 & 500 & 1859 & 10471 & 1309 & 27751 \\
    5 & 500 & 1922 & 10355 & 1256 & 27321 \\
    6 & 500 & 1919 & 10118 & 1302 & 27181 \\\hline
    \textbf{średnia} & \textbf{500} & \textbf{1872} & \textbf{11209} & \textbf{1349} & \textbf{28879} \\
    \hline
    \end{tabular}%
  \label{tab:wall_f1}%
\end{table}%


\begin{figure}
\centering
\includegraphics[width=0.8\textwidth]{pict/mikolajczyk/wall/F1.png}
\caption{WALL - ilość wyszukanych cech}
\label{fig:wall_f1}
\end{figure}


% Table generated by Excel2LaTeX from sheet 'm.wall F'
\begin{table}[htbp]
  \centering
  \caption{WALL - czas lokalizowania pojedynczego punktu charakterystycznego}
    \begin{tabular}{|c|c|c|c|c|c|}
    \hline
    obraz & \textbf{ORB} & \textbf{SIFT} & \textbf{SURF} & \textbf{STAR} & \textbf{FAST} \\
    \hline
    -  & [ms] & [ms] & [ms] & [ms] & [ms] \\\hline
    1 & 0,124 & 0,108 & 0,155 & 0,044 & 0,001 \\
    2 & 0,102 & 0,094 & 0,158 & 0,038 & 0,001 \\
    3 & 0,098 & 0,093 & 0,159 & 0,041 & 0,001 \\
    4 & 0,104 & 0,092 & 0,159 & 0,042 & 0,001 \\
    5 & 0,102 & 0,089 & 0,160 & 0,044 & 0,001 \\
    6 & 0,104 & 0,090 & 0,160 & 0,042 & 0,001 \\\hline
    \textbf{średnia} & \textbf{0,106} & \textbf{0,094} & \textbf{0,158} & \textbf{0,042} & \textbf{0,001} \\
    \hline
    \end{tabular}%
  \label{tab:wall_f2}%
\end{table}%


\begin{figure}
\centering
\includegraphics[width=0.8\textwidth]{pict/mikolajczyk/wall/f2.png}
\caption{WALL - czas lokalizowania pojedynczego punktu charakterystycznego}
\label{fig:wall_f2}
\end{figure}

% Table generated by Excel2LaTeX from sheet 'm.wall F'
\begin{table}[htbp]
  \centering
  \caption{WALL - czas generowania pojedynczego deskryptora punktu charakterystycznego}
    \begin{tabular}{|c|c|c|c|c|c|c|c|}\hline

    obraz & \textbf{ORB} & \textbf{SIFT} & \textbf{SURF} & \textbf{ST-BRIEF} & \textbf{ST-ORB} & \textbf{ST-SIFT} & \textbf{ST-SURF} \\\hline

    - & [ms] & [ms] & [ms] & [ms] & [ms] & [ms] & [ms] \\\hline
    1 & 0,064 & 0,506 & 0,230 & 0,012 & 0,014 & 0,272 & 0,081 \\
    2 & 0,054 & 0,491 & 0,236 & 0,012 & 0,013 & 0,262 & 0,080 \\
    3 & 0,054 & 0,495 & 0,233 & 0,012 & 0,013 & 0,271 & 0,080 \\
    4 & 0,054 & 0,496 & 0,224 & 0,012 & 0,013 & 0,286 & 0,082 \\
    5 & 0,056 & 0,495 & 0,227 & 0,012 & 0,014 & 0,312 & 0,082 \\
    6 & 0,056 & 0,489 & 0,216 & 0,012 & 0,013 & 0,323 & 0,082 \\\hline
    \textbf{średnia} & \textbf{0,056} & \textbf{0,495} & \textbf{0,227} & \textbf{0,012} & \textbf{0,013} & \textbf{0,288} & \textbf{0,081} \\\hline
    

    \end{tabular}%
  \label{tab:wall_f3}%
\end{table}%


\begin{figure}
\centering
\includegraphics[width=0.8\textwidth]{pict/mikolajczyk/wall/f3.png}
\caption{WALL - czas generowania pojedynczego deskryptora punktu charakterystycznego}
\label{fig:wall_f3}
\end{figure}


% Table generated by Excel2LaTeX from sheet 'm.wall M'
\begin{table}[htbp]
  \centering
  \caption{WALL - powtarzalność wykrywanych cech}
    \begin{tabular}{|c|c|c|c|c|c|c|c|}\hline

    obrazy & \textbf{ORB} & \textbf{SIFT} & \textbf{SURF} & \textbf{ST-BRIEF} & \textbf{ST-ORB} & \textbf{ST-SIFT} & \textbf{ST-SURF} \\\hline

    -  & [\%] & [\%] & [\%] & [\%] & [\%] & [\%] & [\%] \\\hline
    1|2 & 32 & 47 & 27 & 59 & 32 & 61 & 6 \\
    1|3 & 22 & 39 & 19 & 43 & 16 & 47 & 2 \\
    1|4 & 7 & 20 & 8 & 23 & 8 & 28 & 1 \\
    1|5 & 1 & 6 & 2 & 9 & 3 & 13 & 1 \\
    1|6 & 1 & 0 & 0 & 1 & 2 & 1 & 1 \\\hline
    \textbf{średnia} & \textbf{13} & \textbf{22} & \textbf{11} & \textbf{27} & \textbf{12} & \textbf{30} & \textbf{2} \\\hline
    

    \end{tabular}%
  \label{tab:wall_m1}%
\end{table}%


\begin{figure}
\centering
\includegraphics[width=0.8\textwidth]{pict/mikolajczyk/wall/m1.png}
\caption{WALL - powtarzalność wykrywanych cech}
\label{fig:wall_m1}
\end{figure}

% Table generated by Excel2LaTeX from sheet 'm.wall M'
\begin{table}[htbp]
  \centering
  \caption{WALL - procent poprawnych dopasowań}
    \begin{tabular}{|c|c|c|c|c|c|c|c|}\hline
    obrazy & \textbf{ORB} & \textbf{SIFT} & \textbf{SURF} & \textbf{ST-BRIEF} & \textbf{ST-ORB} & \textbf{ST-SIFT} & \textbf{ST-SURF} \\\hline
    1|2 & 65 & 89 & 66 & 71 & 89 & 83 & 61 \\
    1|3 & 65 & 76 & 65 & 62 & 69 & 80 & 28 \\
    1|4 & 51 & 68 & 56 & 61 & 45 & 63 & 16 \\
    1|5 & 50 & 43 & 28 & 50 & 18 & 55 & 10 \\
    1|6 & 67 & 25 & 5 & 25 & 10 & 52 & 8 \\\hline
    \textbf{średnia} & \textbf{59} & \textbf{60} & \textbf{44} & \textbf{53} & \textbf{46} & \textbf{67} & \textbf{25} \\\hline
    
    \end{tabular}%
  \label{tab:wall_m2}%
\end{table}%


\begin{figure}
\centering
\includegraphics[width=0.8\textwidth]{pict/mikolajczyk/wall/m2.png}
\caption{WALL - procent poprawnych dopasowań}
\label{fig:wall_m2}
\end{figure}
\FloatBarrier
\subsection{Dyskusja wyników}
W przypadku zmiany położenia obserwatora algorytmy SURF i FAST górują wyraźnie nad pozostałymi w ilości lokalizowanych punktów charakterystycznych. Dla reszty algorytmów ilość lokalizowanych cech pozostaje na względnie stałym poziomie. Ze względu na rożną charakterystykę zbiorów WALL i GRAFFITI ciężko mówić o ujednoliconym trendzie zależnym od zmiany położenia.

Stały czas lokalizownaia niezalezny od scenerii ma algorytm SURF. Należy jednak zwrócić uwage, że algorytm ten wskaźnik ten ma najwyższy spośród badanych metod. W przypadku pozostałych algorytmów wahania czasu rzadko przekraczają $0,02$  milisekundy, co pozwala nierozpatrywać tych zmian jako bardzo istotnych. Z grupy algorytmów wykorzystywanych w badaniu dopasowań najlepsze wyniki osiąga algorytm STAR, algortym SIFT i ORB osiągają zbliżone wyniki.

Najkorzystniejszy współczynnik czasu generowania deskryptora osiąga algorytm BRIEF i jego ewolucja w postaci algorytmu ORB. Wyjątkową niestabilnością charakteryzuje się kombinacja algorytmów STAR - SIFT. Metoda ta jest bardzo zależna od badanej sceny. Podobnie jednak jak w przypadku rozmycia obrazów zestawienie tych algorytmów wykazuje się najodporniejszym współczynnikiem powtarzalności w obrazach o "chropowatej" teksturze. Trzeba jednakże zaznaczyć, że wszystkie algorytmy są bardzo wrażliwe na zmiane położenia obserwatora co skutkuje ogromnym spadkiem powtarzalności.

W przypadku zmiany położenia obserwatora w w znaczący sposób spada również procent trafnych dopasowań. Z analizy wyników dla scen WALL i GRAFFITI, widać, że algortmy lepiej radzą sobie dla zbiorów o "chropowatej" teksturze. Spowodowane to jest faktem, że większość metod opiera swe działanie o detekcje rogów. Na tym tle najlepiej wypadaja algorytmy ORB, SIFT i STAR-SIFT.







\FloatBarrier

\section{Rotacja i zmiana skali}
\FloatBarrier
\subsection{BARK}

\begin{figure}[!htb]
\begin{center}

\subfigure[BARK 1]{
\includegraphics[width=5cm]{pict/mikolajczyk/bark/img1.jpg}
}
\subfigure[BARK 2]{
\includegraphics[width=5cm]{pict/mikolajczyk/bark/img2.jpg}
}
\subfigure[BARK 3]{
\includegraphics[width=5cm]{pict/mikolajczyk/bark/img3.jpg}
}
\subfigure[BARK 4]{
\includegraphics[width=5cm]{pict/mikolajczyk/bark/img4.jpg}
}
\subfigure[BARK 5]{
\includegraphics[width=5cm]{pict/mikolajczyk/bark/img5.jpg}
}
\subfigure[BARK 6]{
\includegraphics[width=5cm]{pict/mikolajczyk/bark/img6.jpg}
}
\caption{BARK - zbiór obrazów testowych}



\label{fig:bark_set}
\end{center}
\end{figure}


% Table generated by Excel2LaTeX from sheet 'm.bark F'
\begin{table}[htbp]
  \centering
  \caption{BARK - ilość wyszukanych cech}
    \begin{tabular}{|c|r|r|r|r|r|}\hline
    
    obraz & \textbf{ORB} & \textbf{SIFT} & \textbf{SURF} & \textbf{STAR} & \textbf{FAST} \\\hline
    
    1 & 500 & 1942 & 4958 & 892 & 15151 \\
    2 & 500 & 1986 & 4809 & 784 & 13496 \\
    3 & 500 & 1729 & 5482 & 1015 & 17422 \\
    4 & 500 & 1672 & 5773 & 1014 & 18678 \\
    5 & 500 & 1593 & 5887 & 854 & 18524 \\
    6 & 500 & 1643 & 6226 & 1063 & 20723 \\\hline
    \textbf{srednia} & \textbf{500} & \textbf{1761} & \textbf{5523} & \textbf{937} & \textbf{17332} \\\hline
    
    \end{tabular}%
  \label{tab:bark_f1}%
\end{table}%


\begin{figure}
\centering
\includegraphics[width=0.8\textwidth]{pict/mikolajczyk/bark/F1.png}
\caption{BARK - ilość wyszukanych cech}
\label{fig:bark_f1}
\end{figure}


% Table generated by Excel2LaTeX from sheet 'm.bark F'
\begin{table}[htbp]
  \centering
  \caption{BARK - czas lokalizowania pojedynczego punktu charakterystycznego}
    \begin{tabular}{|c|c|c|c|c|c|}
    \hline
    obraz & \textbf{ORB} & \textbf{SIFT} & \textbf{SURF} & \textbf{STAR} & \textbf{FAST} \\
    \hline
    -  & [ms] & [ms] & [ms] & [ms] & [ms] \\\hline
    1 & 0,096 & 0,065 & 0,169 & 0,037 & 0,001 \\
    2 & 0,082 & 0,064 & 0,171 & 0,040 & 0,001 \\
    3 & 0,098 & 0,071 & 0,166 & 0,032 & 0,001 \\
    4 & 0,102 & 0,074 & 0,163 & 0,032 & 0,001 \\
    5 & 0,098 & 0,076 & 0,163 & 0,036 & 0,001 \\
    6 & 0,110 & 0,074 & 0,162 & 0,030 & 0,001 \\\hline
    \textbf{średnia} & \textbf{0,098} & \textbf{0,071} & \textbf{0,166} & \textbf{0,034} & \textbf{0,001} \\\hline
    \end{tabular}%
  \label{tab:bark_f2}%
\end{table}%


\begin{figure}
\centering
\includegraphics[width=0.8\textwidth]{pict/mikolajczyk/bark/f2.png}
\caption{BARK - czas lokalizowania pojedynczego punktu charakterystycznego}
\label{fig:bark_f2}
\end{figure}

% Table generated by Excel2LaTeX from sheet 'm.bark F'
\begin{table}[htbp]
  \centering
  \caption{BARK - czas generowania pojedynczego deskryptora punktu charakterystycznego}
    \begin{tabular}{|c|c|c|c|c|c|c|c|}\hline

    obraz & \textbf{ORB} & \textbf{SIFT} & \textbf{SURF} & \textbf{ST-BRIEF} & \textbf{ST-ORB} & \textbf{ST-SIFT} & \textbf{ST-SURF} \\\hline

    - & [ms] & [ms] & [ms] & [ms] & [ms] & [ms] & [ms] \\\hline
    1 & 0,056 & 0,216 & 0,199 & 0,012 & 0,012 & 0,855 & 0,092 \\
    2 & 0,056 & 0,207 & 0,205 & 0,011 & 0,013 & 0,810 & 0,091 \\
    3 & 0,056 & 0,211 & 0,205 & 0,012 & 0,013 & 0,710 & 0,089 \\
    4 & 0,058 & 0,217 & 0,199 & 0,012 & 0,013 & 0,624 & 0,086 \\
    5 & 0,056 & 0,220 & 0,202 & 0,012 & 0,014 & 0,562 & 0,087 \\
    6 & 0,056 & 0,216 & 0,201 & 0,012 & 0,012 & 0,568 & 0,087 \\\hline
    \textbf{średnia} & \textbf{0,056} & \textbf{0,215} & \textbf{0,202} & \textbf{0,012} & \textbf{0,013} & \textbf{0,688} & \textbf{0,089} \\\hline

    \end{tabular}%
  \label{tab:bark_f3}%
\end{table}%


\begin{figure}
\centering
\includegraphics[width=0.8\textwidth]{pict/mikolajczyk/bark/f3.png}
\caption{BARK - czas generowania pojedynczego deskryptora punktu charakterystycznego}
\label{fig:bark_f3}
\end{figure}


% Table generated by Excel2LaTeX from sheet 'm.bark M'
\begin{table}[htbp]
  \centering
  \caption{BARK - powtarzalność wykrywanych cech}
    \begin{tabular}{|c|c|c|c|c|c|c|c|}\hline

    obrazy & \textbf{ORB} & \textbf{SIFT} & \textbf{SURF} & \textbf{ST-BRIEF} & \textbf{ST-ORB} & \textbf{ST-SIFT} & \textbf{ST-SURF} \\\hline

    -  & [\%] & [\%] & [\%] & [\%] & [\%] & [\%] & [\%] \\\hline
    1|2 & 15 & 19 & 10 & 2 & 3 & 0 & 2 \\
    1|3 & 8 & 10 & 6 & 1 & 2 & 0 & 2 \\
    1|4 & 6 & 11 & 5 & 2 & 1 & 0 & 2 \\
    1|5 & 3 & 8 & 2 & 1 & 2 & 0 & 2 \\
    1|6 & 1 & 4 & 2 & 1 & 2 & 0 & 2 \\\hline
    \textbf{średnia} & \textbf{6} & \textbf{10} & \textbf{5} & \textbf{1} & \textbf{2} & \textbf{0} & \textbf{2} \\\hline

    \end{tabular}%
  \label{tab:bark_m1}%
\end{table}%


\begin{figure}
\centering
\includegraphics[width=0.8\textwidth]{pict/mikolajczyk/bark/m1.png}
\caption{BARK - powtarzalność wykrywanych cech}
\label{fig:bark_m1}
\end{figure}

% Table generated by Excel2LaTeX from sheet 'm.bark M'
\begin{table}[htbp]
  \centering
  \caption{BARK - procent poprawnych dopasowań}
    \begin{tabular}{|c|c|c|c|c|c|c|c|}\hline
    obrazy & \textbf{ORB} & \textbf{SIFT} & \textbf{SURF} & \textbf{ST-BRIEF} & \textbf{ST-ORB} & \textbf{ST-SIFT} & \textbf{ST-SURF} \\\hline
     - & [\%] & [\%] & [\%] & [\%] & [\%] & [\%] & [\%] \\\hline
    1|2 & 49 & 50 & 46 & 24 & 12 & 0 & 18 \\
    1|3 & 33 & 43 & 38 & 21 & 18 & 0 & 22 \\
    1|4 & 44 & 80 & 28 & 19 & 19 & 0 & 20 \\
    1|5 & 36 & 78 & 14 & 21 & 14 & 0 & 23 \\
    1|6 & 0 & 71 & 12 & 19 & 18 & 0 & 8 \\\hline
    \textbf{średnia} & \textbf{33} & \textbf{64} & \textbf{28} & \textbf{21} & \textbf{16} & \textbf{0} & \textbf{18} \\\hline
    \end{tabular}%
  \label{tab:bark_m2}%
\end{table}%


\begin{figure}
\centering
\includegraphics[width=0.8\textwidth]{pict/mikolajczyk/bark/m2.png}
\caption{BARK - procent poprawnych dopasowań}
\label{fig:bark_m2}
\end{figure}






\FloatBarrier
\subsection{BOAT}

\begin{figure}[!htb]
\begin{center}
\subfigure[BOAT 1]{
\includegraphics[width=5cm]{pict/mikolajczyk/boat/img1.jpg}
}
\subfigure[BOAT 2]{
\includegraphics[width=5cm]{pict/mikolajczyk/boat/img2.jpg}
}
\subfigure[BOAT 3]{
\includegraphics[width=5cm]{pict/mikolajczyk/boat/img3.jpg}
}
\subfigure[BOAT 4]{
\includegraphics[width=5cm]{pict/mikolajczyk/boat/img4.jpg}
}
\subfigure[BOAT 5]{
\includegraphics[width=5cm]{pict/mikolajczyk/boat/img5.jpg}
}
\subfigure[BOAT 6]{
\includegraphics[width=5cm]{pict/mikolajczyk/boat/img6.jpg}
}
\caption{BOAT - zbiór obrazów testowych}
\label{fig:boat_set}
\end{center}
\end{figure}



% Table generated by Excel2LaTeX from sheet 'm.boat F'
\begin{table}[htbp]
  \centering
  \caption{BOAT - ilość wyszukanych cech}
    \begin{tabular}{|c|r|r|r|r|r|}\hline
    
    obraz & \textbf{ORB} & \textbf{SIFT} & \textbf{SURF} & \textbf{STAR} & \textbf{FAST} \\\hline
    
   1 & 500 & 1451 & 10302 & 1923 & 18122 \\
    2 & 500 & 1352 & 11467 & 1853 & 19076 \\
    3 & 500 & 1238 & 10375 & 1306 & 16222 \\
    4 & 500 & 727 & 11912 & 924 & 13377 \\
    5 & 500 & 726 & 11274 & 896 & 10668 \\
    6 & 500 & 687 & 11099 & 658 & 13394 \\\hline
    \textbf{średnia} & \textbf{500} & \textbf{1030} & \textbf{11072} & \textbf{1260} & \textbf{15143} \\
   \hline
    \end{tabular}%
  \label{tab:boat_f1}%
\end{table}%


\begin{figure}
\centering
\includegraphics[width=0.8\textwidth]{pict/mikolajczyk/boat/F1.png}
\caption{BOAT - ilość wyszukanych cech}
\label{fig:boat_f1}
\end{figure}


% Table generated by Excel2LaTeX from sheet 'm.boat F'
\begin{table}[htbp]
  \centering
  \caption{BOAT - czas lokalizowania pojedynczego punktu charakterystycznego}
    \begin{tabular}{|c|c|c|c|c|c|}
    \hline
    obraz & \textbf{ORB} & \textbf{SIFT} & \textbf{SURF} & \textbf{STAR} & \textbf{FAST} \\
    \hline
    -  & [ms] & [ms] & [ms] & [ms] & [ms] \\\hline
   1 & 0,120 & 0,112 & 0,158 & 0,029 & 0,001 \\
    2 & 0,124 & 0,118 & 0,155 & 0,030 & 0,001 \\
    3 & 0,112 & 0,124 & 0,156 & 0,041 & 0,001 \\
    4 & 0,090 & 0,198 & 0,152 & 0,057 & 0,001 \\
    5 & 0,078 & 0,196 & 0,155 & 0,059 & 0,001 \\
    6 & 0,088 & 0,205 & 0,154 & 0,079 & 0,001 \\\hline
    \textbf{średnia} & \textbf{0,102} & \textbf{0,159} & \textbf{0,155} & \textbf{0,049} & \textbf{0,001} \\
   \hline
    \end{tabular}%
  \label{tab:boat_f2}%
\end{table}%


\begin{figure}
\centering
\includegraphics[width=0.8\textwidth]{pict/mikolajczyk/boat/f2.png}
\caption{BOAT - czas lokalizowania pojedynczego punktu charakterystycznego}
\label{fig:boat_f2}
\end{figure}

% Table generated by Excel2LaTeX from sheet 'm.boat F'
\begin{table}[htbp]
  \centering
  \caption{BOAT - czas generowania pojedynczego deskryptora punktu charakterystycznego}
    \begin{tabular}{|c|c|c|c|c|c|c|c|}\hline

    obraz & \textbf{ORB} & \textbf{SIFT} & \textbf{SURF} & \textbf{ST-BRIEF} & \textbf{ST-ORB} & \textbf{ST-SIFT} & \textbf{ST-SURF} \\\hline

    - & [ms] & [ms] & [ms] & [ms] & [ms] & [ms] & [ms] \\\hline
    1 & 0,082 & 0,241 & 0,196 & 0,011 & 0,013 & 0,996 & 0,093 \\
    2 & 0,082 & 0,249 & 0,195 & 0,012 & 0,013 & 0,887 & 0,091 \\
    3 & 0,084 & 0,254 & 0,183 & 0,012 & 0,014 & 0,969 & 0,093 \\
    4 & 0,084 & 0,315 & 0,249 & 0,013 & 0,015 & 1,049 & 0,093 \\
    5 & 0,082 & 0,320 & 0,202 & 0,012 & 0,016 & 1,103 & 0,094 \\
    6 & 0,084 & 0,332 & 0,185 & 0,014 & 0,017 & 1,195 & 0,096 \\\hline
    \textbf{średnia} & \textbf{0,083} & \textbf{0,285} & \textbf{0,202} & \textbf{0,012} & \textbf{0,015} & \textbf{1,033} & \textbf{0,093} \\\hline
    
    \end{tabular}%
  \label{tab:boat_f3}%
\end{table}%


\begin{figure}
\centering
\includegraphics[width=0.8\textwidth]{pict/mikolajczyk/boat/f3.png}
\caption{BOAT - czas generowania pojedynczego deskryptora punktu charakterystycznego}
\label{fig:boat_f3}
\end{figure}


% Table generated by Excel2LaTeX from sheet 'm.boat M'
\begin{table}[htbp]
  \centering
  \caption{BOAT - powtarzalność wykrywanych cech}
    \begin{tabular}{|c|c|c|c|c|c|c|c|}\hline

    obrazy & \textbf{ORB} & \textbf{SIFT} & \textbf{SURF} & \textbf{ST-BRIEF} & \textbf{ST-ORB} & \textbf{ST-SIFT} & \textbf{ST-SURF} \\\hline

   - & [\%] & [\%] & [\%] & [\%] & [\%] & [\%] & [\%] \\
    1|2 & 28 & 27 & 10 & 10 & 3 & 22 & 6 \\
    1|3 & 29 & 33 & 6 & 1 & 1 & 0 & 7 \\
    1|4 & 17 & 20 & 6 & 1 & 2 & 0 & 3 \\
    1|5 & 11 & 11 & 4 & 2 & 2 & 4 & 2 \\
    1|6 & 3 & 5 & 1 & 2 & 1 & 0 & 2 \\\hline
    \textbf{średnia} & \textbf{18} & \textbf{19} & \textbf{5} & \textbf{3} & \textbf{2} & \textbf{5} & \textbf{4} \\\hline
   
    \end{tabular}%
  \label{tab:boat_m1}%
\end{table}%


\begin{figure}
\centering
\includegraphics[width=0.8\textwidth]{pict/mikolajczyk/boat/m1.png}
\caption{BOAT - powtarzalność wykrywanych cech}
\label{fig:boat_m1}
\end{figure}

% Table generated by Excel2LaTeX from sheet 'm.boat M'
\begin{table}[htbp]
  \centering
  \caption{BOAT - procent poprawnych dopasowań}
    \begin{tabular}{|c|c|c|c|c|c|c|c|}\hline
    obrazy & \textbf{ORB} & \textbf{SIFT} & \textbf{SURF} & \textbf{ST-BRIEF} & \textbf{ST-ORB} & \textbf{ST-SIFT} & \textbf{ST-SURF} \\\hline
     - & [\%] & [\%] & [\%] & [\%] & [\%] & [\%] & [\%] \\\hline
   
    1|2 & 43 & 49 & 32 & 61 & 38 & 56 & 28 \\
    1|3 & 49 & 77 & 19 & 15 & 14 & 0 & 38 \\
    1|4 & 55 & 48 & 29 & 10 & 9 & 100 & 22 \\
    1|5 & 43 & 37 & 18 & 9 & 9 & 15 & 5 \\
    1|6 & 42 & 29 & 3 & 16 & 13 & 0 & 5 \\\hline
    \textbf{średnia} & \textbf{46} & \textbf{48} & \textbf{20} & \textbf{22} & \textbf{16} & \textbf{34} & \textbf{20} \\\hline
     \end{tabular}%
  \label{tab:boat_m2}%
\end{table}%


\begin{figure}
\centering
\includegraphics[width=0.8\textwidth]{pict/mikolajczyk/boat/m2.png}
\caption{BOAT - procent poprawnych dopasowań}
\label{fig:boat_m2}
\end{figure}
\FloatBarrier
\subsection{Dyskusja wyników}
Dla zestawów poddanych rotacji i zmianie skali możemy zaobserwować większość trendów widocznych w poprzednich próbach. Najwieksze wahnięcia w ilości lokalizowanych puntków obserwujemy dla algorytmu FAST. Pozostałe metody lokalizują zbliżoną liczbę punktów w obrębie zestawów.

Średni czas lokalizowania cechy dla większości algorytmów jest stały niezależnie od zbioru. Najstabilniejszy pod tym względem jest algorytm ORB. Wyniki czasowe algorytmu SIFT są najbardziej zróżnicowane. Algorytm ten osiąga jednakże najlepsze wyniki pod względem powtarzalności lokalizowanych cech.

Pod względem czasu generowanie deskryptora zachowana zostaje zachowana klasyfikacja algorytmów jak w przypadku wczesniejszych zbiorów. Tradycyjnie najgorsze wyniki w tym polu osiąga kombinacja STAR-SIFT. O ile w przypadku wcześniejszych zbiorów kombinacja ta osiągała dobre wyniki w obszarze dopasowań, o tyle dla zbiorów BARK i BOAT, algorytm nie działa.

Rotacja połączona ze zmianą skali jest najbardziej wymagającym zbiorem i wyniki osiągane przez wszystkie algorytmy są słabe. Po analizie wyników zdecydowano się w przypadku badań obszarów skalnych zbiór ten podzielić na dwa, reprezentujące jeden typ przekształcenia.



\FloatBarrier
\section{Zmiana oświetlenia}
\subsection{CARS}

\begin{figure}[!htb]
\begin{center}
\subfigure[CARS 1]{
\includegraphics[width=5cm]{pict/mikolajczyk/light/img1.jpg}
}
\subfigure[CARS 2]{
\includegraphics[width=5cm]{pict/mikolajczyk/light/img2.jpg}
}
\subfigure[CARS 3]{
\includegraphics[width=5cm]{pict/mikolajczyk/light/img3.jpg}
}
\subfigure[CARS 4]{
\includegraphics[width=5cm]{pict/mikolajczyk/light/img4.jpg}
}
\subfigure[CARS 5]{
\includegraphics[width=5cm]{pict/mikolajczyk/light/img5.jpg}
}
\subfigure[CARS 6]{
\includegraphics[width=5cm]{pict/mikolajczyk/light/img6.jpg}
}
\caption{CARS -Zbiór obrazów testowych}
\
\label{fig:cars_}
\end{center}
\end{figure}



% Table generated by Excel2LaTeX from sheet 'm.cars F'
\begin{table}[htbp]
  \centering
  \caption{CARS -ilość wyszukanych cech}
    \begin{tabular}{|c|r|r|r|r|r|}\hline
    
    obraz & \textbf{ORB} & \textbf{SIFT} & \textbf{SURF} & \textbf{STAR} & \textbf{FAST} \\\hline
    
   
    1 & 500 & 688 & 12062 & 477 & 9431 \\
    2 & 500 & 595 & 12278 & 299 & 7216 \\
    3 & 500 & 515 & 12234 & 253 & 6301 \\
    4 & 500 & 481 & 12141 & 219 & 5806 \\
    5 & 500 & 463 & 12215 & 204 & 5336 \\
    6 & 500 & 445 & 12008 & 190 & 4974 \\\hline
    \textbf{średnia} & \textbf{500} & \textbf{531} & \textbf{12156} & \textbf{274} & \textbf{6511} \\
    \hline
    \end{tabular}%
  \label{tab:cars_f1}%
\end{table}%


\begin{figure}
\centering
\includegraphics[width=0.8\textwidth]{pict/mikolajczyk/light/F1.png}
\caption{CARS -ilość wyszukanych cech}
\label{fig:cars_f1}
\end{figure}


% Table generated by Excel2LaTeX from sheet 'm.cars F'
\begin{table}[htbp]
  \centering
  \caption{CARS - czas lokalizowania pojedynczego punktu charakterystycznego}
    \begin{tabular}{|c|c|c|c|c|c|}
    \hline
    obraz & \textbf{ORB} & \textbf{SIFT} & \textbf{SURF} & \textbf{STAR} & \textbf{FAST} \\
    \hline
    -  & [ms] & [ms] & [ms] & [ms] & [ms] \\\hline
    1 & 0,066 & 0,199 & 0,150 & 0,105 & 0,001 \\
    2 & 0,058 & 0,220 & 0,150 & 0,167 & 0,001 \\
    3 & 0,056 & 0,258 & 0,149 & 0,198 & 0,001 \\
    4 & 0,054 & 0,277 & 0,150 & 0,247 & 0,001 \\
    5 & 0,054 & 0,283 & 0,150 & 0,230 & 0,001 \\
    6 & 0,052 & 0,299 & 0,150 & 0,247 & 0,001 \\\hline
    \textbf{średnia} & \textbf{0,057} & \textbf{0,256} & \textbf{0,150} & \textbf{0,199} & \textbf{0,001} \\
    \hline
    \end{tabular}%
  \label{tab:cars_f2}%
\end{table}%


\begin{figure}
\centering
\includegraphics[width=0.8\textwidth]{pict/mikolajczyk/light/f2.png}
\caption{CARS - czas lokalizowania pojedynczego punktu charakterystycznego}
\label{fig:cars_f2}
\end{figure}

% Table generated by Excel2LaTeX from sheet 'm.cars F'
\begin{table}[htbp]
  \centering
  \caption{CARS - czas generowania pojedynczego deskryptora punktu charakterystycznego}
    \begin{tabular}{|c|c|c|c|c|c|c|c|}\hline

    obraz & \textbf{ORB} & \textbf{SIFT} & \textbf{SURF} & \textbf{ST-BRIEF} & \textbf{ST-ORB} & \textbf{ST-SIFT} & \textbf{ST-SURF} \\\hline

    - & [ms] & [ms] & [ms] & [ms] & [ms] & [ms] & [ms] \\\hline
    1 & 0,072 & 0,320 & 0,178 & 0,015 & 0,019 & 1,537 & 0,101 \\
    2 & 0,072 & 0,343 & 0,180 & 0,017 & 0,023 & 1,722 & 0,104 \\
    3 & 0,070 & 0,367 & 0,181 & 0,020 & 0,028 & 1,846 & 0,107 \\
    4 & 0,072 & 0,391 & 0,182 & 0,018 & 0,032 & 1,968 & 0,105 \\
    5 & 0,072 & 0,389 & 0,183 & 0,025 & 0,029 & 1,917 & 0,108 \\
    6 & 0,072 & 0,409 & 0,185 & 0,021 & 0,032 & 1,900 & 0,105 \\\hline
    \textbf{średnia} & \textbf{0,072} & \textbf{0,370} & \textbf{0,181} & \textbf{0,019} & \textbf{0,027} & \textbf{1,815} & \textbf{0,105} \\\hline
    
    \end{tabular}%
  \label{tab:cars_f3}%
\end{table}%


\begin{figure}
\centering
\includegraphics[width=0.8\textwidth]{pict/mikolajczyk/light/f3.png}
\caption{CARS - czas generowania pojedynczego deskryptora punktu charakterystycznego}
\label{fig:cars_f3}
\end{figure}


% Table generated by Excel2LaTeX from sheet 'm.cars M'
\begin{table}[htbp]
  \centering
  \caption{CARS - powtarzalność wykrywanych cech}
    \begin{tabular}{|c|c|c|c|c|c|c|c|}\hline

    obrazy & \textbf{ORB} & \textbf{SIFT} & \textbf{SURF} & \textbf{ST-BRIEF} & \textbf{ST-ORB} & \textbf{ST-SIFT} & \textbf{ST-SURF} \\\hline

    -  & [\%] & [\%] & [\%] & [\%] & [\%] & [\%] & [\%] \\\hline
    1|2 & 33 & 54 & 39 & 51 & 51 & 53 & 25 \\
    1|3 & 20 & 48 & 34 & 46 & 44 & 48 & 21 \\
    1|4 & 17 & 41 & 34 & 39 & 39 & 40 & 20 \\
    1|5 & 11 & 38 & 31 & 40 & 36 & 37 & 18 \\
    1|6 & 7 & 36 & 22 & 31 & 30 & 31 & 11 \\\hline
    \textbf{średnia} & \textbf{18} & \textbf{43} & \textbf{32} & \textbf{41} & \textbf{40} & \textbf{42} & \textbf{19} \\\hline
    
    \end{tabular}%
  \label{tab:cars_m1}%
\end{table}%


\begin{figure}
\centering
\includegraphics[width=0.8\textwidth]{pict/mikolajczyk/light/m1.png}
\caption{CARS - powtarzalność wykrywanych cech}
\label{fig:cars_m1}
\end{figure}

% Table generated by Excel2LaTeX from sheet 'm.cars M'
\begin{table}[htbp]
  \centering
  \caption{CARS - procent poprawnych dopasowań}
    \begin{tabular}{|c|c|c|c|c|c|c|c|}\hline
    obrazy & \textbf{ORB} & \textbf{SIFT} & \textbf{SURF} & \textbf{ST-BRIEF} & \textbf{ST-ORB} & \textbf{ST-SIFT} & \textbf{ST-SURF} \\\hline
     - & [\%] & [\%] & [\%] & [\%] & [\%] & [\%] & [\%] \\\hline
    1|2 & 72 & 83 & 59 & 71 & 80 & 80 & 56 \\
    1|3 & 66 & 81 & 68 & 61 & 71 & 58 & 51 \\
    1|4 & 61 & 66 & 63 & 60 & 63 & 62 & 50 \\
    1|5 & 53 & 67 & 69 & 49 & 67 & 55 & 43 \\
    1|6 & 35 & 69 & 49 & 57 & 49 & 59 & 31 \\\hline
    \textbf{średnia} & \textbf{57} & \textbf{73} & \textbf{62} & \textbf{60} & \textbf{66} & \textbf{63} & \textbf{46} \\\hline
   
    \end{tabular}%
  \label{tab:cars_m2}%
\end{table}%


\begin{figure}
\centering
\includegraphics[width=0.8\textwidth]{pict/mikolajczyk/light/m2.png}
\caption{CARS - procent poprawnych dopasowań}
\label{fig:cars_m2}
\end{figure}


\FloatBarrier
\subsection{Dyskusja wyników}
Większość algorytmów badanych w tym zbiorze wykazuje wrażliwość na stopień oświetlenia. Przejawia się to spadkiem ilości lokalizowanych cech. Spadek ten jest szczególnie widoczny w przypadku algorytmu FAST. Wyjątkiem w tym zestawieniu jest algorytm SURF, którego ilość lokalizowanych cech wydaje się być niezależna od oświetlenia.

Podobnie sytuacja kształtuje się w obszarze charakterystyk czasowych. O ile większość algorytmów wraz z przyciemnieniem zdjęcia potrzebuje więcej czasu na zlokalizowanie punktu, SURF ma stały czas wykonania zadania.

Czas generowanie deskryptorów w obrębie zbioru wydaje się być stabilny dla wszystkich algorytmów. Najlepsze wyniki w tym obszarze osiągają kombinacje algorytmów STAR z deskryptorami BRIEF i ORB oraz kompletny algorytm ORB. Ponownie możemy zaobserwować długi czas generowania deskryptora STAR-SIFT.

Najwyższy procent powtarzalności posiada algorytm SIFT. Wolna kombinacja STAR-SIFT również osiąga wysokie parametry powtarzalności. Dobre wskaźniki posiadają również algorytmy SURF ,STAR-BRIEF i STAR-ORB.

Wspomniane algorytmy również osiągają zadowalające poprawności dopasowań. Najsłabiej pod tym względem wypadają algorytmy ORB i STAR-SURF.




\FloatBarrier